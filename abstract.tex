\thispagestyle{fancy}
% Leave Left and Right Header empty.
\lhead{}
\rhead{}
\renewcommand{\headrulewidth}{1pt}
\renewcommand{\footrulewidth}{0pt}
\newcommand{\packageName}{\textit{SNTD}}
\renewcommand{\baselinestretch}{1.5} 

\fancyfoot[C]{}

\pagestyle{fancy}
\lhead{\bf NASA EPSCoR GRA $\|$ \ \ }
\rhead{\bf Justin D. Roberts-Pierel, University of South Carolina}
\setstretch{1.09}
Observing and understanding gravitationally lensed objects is a new
and exciting field of study. The first examples of a strongly lensed
supernova (SN) resolved into multiple images was SN ``Refsdal'' in
2014 (Figure 1; \citet{Kelly:2015a}). Subsequent work was done to
classify Refsdal, as well as to measure time delays and magnification
ratios \citep{Kelly:2016,Rodney:2016}. Another lensed Type 1a SN was found
by \cite{Goobar:2016}, and \textbf{the next decade is expected to yield observations
of over 100 lensed SNe that will require analysis} (\cite{Oguri:2010}).
To date, there is no public software package for analyzing multiply-imaged SNe. 
The lack of a standard resource leaves researchers to write and implement their 
own ad hoc programs, which will become increasingly inefficient as the number of
observed lensed SNe increases. 
An optimized software package with the ability to analyze a light curve for 
these and other parameters would be extremely valuable to a wide range of
 research using current and next generation telescopes.

\begin{figure}[h]
\centering
\includegraphics[width=0.9\textwidth]{refsdal_rodney.pdf}
\caption{
MACS J1149.6+2223 field, showing the positions of the three primary
images of the SN Refsdal host galaxy (labeled 1.1, 1.2, and 1.3). SN
Refsdal appears as four point sources in an Einstein Cross
configuration in the southeast spiral arm of image 1.1 \cite{Rodney:2016}}
\end{figure}


The purpose of this work is to produce an open-source software
package called Supernova Time Delays ($\packageName$). \textbf{This package will
lead to an improved understanding of the complex effects acting on 
the two known multiply-imaged SNe, and prepare for the need to analyze 
lensed SNe over the next decade}.
There are currently two software packages that form the basis 
of this product: Python Curve Shifting (PyCS), and Supernova Cosmology 
(SNCosmo). There will be four components to this research: 1) Integrate 
SNCosmo and PyCS, giving researchers a tool
that can model SN light curve data with the abilities present in either
software package; 2) Extend and optimize the lensing and microlensing
algorithm present in PyCS for SNe; 3) Simulate a large number of multiply-imaged SN
light curves using SNCosmo and test the ability of $\packageName$ to
determine SN parameters; and 4) Explore the abilities of
$\packageName$ in analyzing POP III SNe using modeled light curves in
anticipation of JWST and WFIRST observations.
%\begin{figure}[h]
%\centering
%\includegraphics[scale=.38]{gra_2017_Flow2.pdf}
%\caption{A visual representation of the capabilities of SNTD. The python package will have a simple framework in which each module creates a respective object, which can then be edited by the user. This structure is extremely organized, computationally efficient, and user-friendly.}
%\end{figure}


While PyCS currently uses flexible functions to account for
microlensing of quasars caused by transverse motion of stars in the
lensing galaxy, there is a second form of microlensing unique to
lensed SNe. The expanding photosphere of a SN interacts with an 
increasing volume of the caustic web, precipitating microlensing effects 
described by \cite{Dobler:2006} that can exhibit magnitude 
fluctuations of $\sim$0.2 to $>$0.5 on timescales of weeks to months. 
This proposed work has already integrated the flexible function method 
(Figure 2), and will seek to include both types of microlensing effects to 
increase the precision and accuracy of time delay and magnification measurements. 
\textbf{No software package currently
exists with this capability}, therefore the algorithm will be developed
following the methodology of \cite{Dobler:2006}. In general,
$\packageName$ will be much more likely to correctly identify SN
microlensing effects and produce simpler time delay measurements
because, unlike for quasars, researchers already have a robust set of
intrinsic light curve shapes to describe a range of different SN
classes that will allow for physically motivated, less flexible
models. By allowing certain constraints to float as free parameters,
these models will also produce best-fit synthetic light curves from
which time delay, magnifications, SN class, and redshift can be
derived. By simulating large numbers of SNe with these synthetic light
curves, we will be able to quantify the accuracy and efficiency of the
software.


%{\setlength{\parindent}{.5cm}

%}
%\begin{figure}[h]
%\centering
%\includegraphics[scale=.4]{sncosmo.png}
%\caption{
%In addition to spline fitting (Figure 2), $\packageName$ is able to
%fit model templates to SN light curve data to measure various
%properties. In future work, the Dobler and Keeton 2006a methodology
%will be used to extract microlensing effects by identifying
%perturbations from these templates.}
%\end{figure}


 \textbf{The era of JWST will make it possible for the first time to detect and study SNe 
 at} $\mathbf{z>5}$ (\cite{Mesinger:2006}), and WFIRST will add many more discoveries at lower redshift 
 with a larger field of view in the mid-2020s. 
In addition, recent simulations of pair-instability (PI) SN light curves indicate
that various solar mass PI and POP III Type IIn SNe will be visible to
both JWST (z$>$30 and z$\sim$15 respectively) and WFIRST (z$\sim$20
and z$\sim$5 respectively) (\cite{Magg:2016}). Whether it is a follow-up observation
of a ground based detection, or a discovery made by one of these space
telescopes, \textbf{having flexible software in place to analyze any SN light
curve will be essential to SN cosmologists.} We expect this research to
yield \textbf{multiple publications} about the methodology of the software
itself and the simulation results for low and high redshift SNe, as
well as advancements in our understanding of lensed SNe resulting from
the use of $\packageName$. There are also plans to use this package
for follow-up measurements of the two known multiply-imaged SN
parameters, including the previously ignored microlensing
effects. $\packageName$ will be written in Python, a free and
\textbf{open-source} programming language, and distributed to SN researchers
with extensive documentation including descriptive tutorials. This
combination will make $\packageName$ a \textbf{free, widely accessible, and
invaluable tool} in the SN research community.  
\begin{figure}[h]

\centering
\includegraphics[width=.74\textwidth]{points_plot_2017.pdf}
\includegraphics[width=.74\textwidth]{refs_plot_2017.pdf}
\includegraphics[width=.74\textwidth]{spline_plot_2017.pdf}
\caption{(Top) HST F160W data representing the four images of SN
Refsdal (Figure 1), with no lensing or time shifts. (Middle) Method of
fitting the SN Refsdal light curves from Rodney et al. 2016, which did
not consider microlensing effects. (Bottom) Preliminary results from 
$\packageName$ using a multi-knot spline to fit the data. This method 
includes microlensing effects, which leads to a slight adjustment in 
time delay measurements. }
\end{figure}

\pagebreak
\bibliographystyle{apjbrief}

\bibliography{bibdesk}

%\


%1. Kelly, Patrick L, Steven A Rodney, Tommaso Treu, Ryan J Foley,
%Gabriel Brammer, Kasper B Schmidt, Adi Zitrin, et
%al. 2015. ``Multiple Images of a Highly Magnified Supernova Formed by
%an Early-Type Cluster Galaxy Lens.'' Science 347 (6226):
%1123-26. doi:10.1126/science.aaa3350.

%\

%2. Rodney, S. A., L.G. Strolger, P. L. Kelly, M. Bradac, G. Brammer,
%A. V. Filippenko, R. J. Foley, et al. 2016. ``SN Refsdal: Photometry
%and Time Delay Measurements of the First Einstein Cross Supernova''
%The Astrophysical Journal 820 (1):
%50. doi:10.3847/0004-637X/820/1/50.

%\

%3. Dobler, Gregory, and Charles R Keeton. 2006. ``Microlensing of
%Lensed Supernovae.'' Astrophysical Journal.

%\

%4. Mesinger, Andrei, and Benjamin D Johnson. 2006. ``The Redshift
%Distribution of Distant Supernovae and Its Use in Probing
%Reionization'', 80-90.

%\

%5. Magg, Mattis, Tilman Hartwig, Simon C O Glover, Ralf S Klessen,
%and J Whalen. 2016. ``A New Statistical Model for Population III
%Supernova Rates?: Discriminating Between $\Lambda$ CDM and WDM
%Cosmologies'' 12 (June): 1-12.












